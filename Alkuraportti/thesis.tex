% $Name:  $
% $Id: thesis.tex,v 1.18 2010/10/08 13:39:40 paalanen Exp $


% The history of this template:
% - unknown, original version
% - Jarmo "trewas" Ilonen, masters thesis, 2003
% - Pekka "PQ" Paalanen, information processing project, 2004,
%     hints about graphicx and making PDF from Pasi Valminen
% - Pekka "PQ" Paalanen, Master's thesis, 2006
% - upgraded to pdflatex and 1.8.2010 thesis guidelines, 2010

% useful links:
% http://www.ctan.org/tex-archive/help/Catalogue/entries/grfguide.html
% http://www.tug.org/applications/hyperref/


\documentclass{lutmscthesis}[2010/09/22]
%\documentclass[draft]{lutmscthesis}   % leave figures blank, faster

\usepackage[latin1]{inputenc}
\usepackage[T1]{fontenc}
\usepackage[english,finnish]{babel}

\usepackage{times}

\usepackage{setspace}
\usepackage{verbatim}
\usepackage[intlimits]{amsmath}

% Ensure figure captions are below and table captions are above the content.
\usepackage{float}
\floatstyle{plain}\restylefloat{figure}
\floatstyle{plaintop}\restylefloat{table}

\usepackage[pdfborder={0 0 0}]{hyperref}
%\usepackage[chapter]{algorithm}

%           Hyperref rationale - or just pain in the butt
%
% Load 'float' package first, because that will fix problems with 'algorithm'
% package interacting with hyperref.
%
% Hyperref must be the last package loaded, except...
% Load 'algorithm' AFTER hyperref, otherwise \theHalgorithm is
% undefined control sequence error appears.
%
% The TeXLive 2008 version of 'algorithmic' is buggy with hyperref.
% Use this bundled, special, hand-fixed version of algorithmic.sty
% instead. It is identified by version 2006/12/15.


\graphicspath{{resources/}}                % Graphics search path

\newcommand{\vect}[1]{\boldsymbol{#1}}
\newcommand{\matr}[1]{\boldsymbol{#1}}
\newcommand{\diag}[1]{\mathrm{diag}(#1)}
\newcommand{\iprod}[1]{\left\langle #1 \right\rangle}
\newcommand{\me}{\mathrm{e}}
\newcommand{\mi}{\mathrm{i}}
\newcommand{\md}{\mathrm{d}}
\newcommand{\sse}{{}} %\mathrm{SSE}}
\newcommand{\trace}{\mathrm{Tr}\:}
\newcommand{\frp}[2]{{}^\mathrm{#1}\vect{#2}}
\newcommand{\frs}[3]{{}^\mathrm{#1}#2_\mathrm{#3}}
\newcommand{\frv}[3]{{}^\mathrm{#1}\vect{#2}_\mathrm{#3}}
\newcommand{\frm}[3]{{}^\mathrm{#1}\matr{#2}_\mathrm{#3}}
\newcommand{\colvec}[2]{\genfrac{[}{]}{0pt}{1}{#1}{#2}}
\newcommand{\relphantom}[1]{\mathrel{\phantom{#1}}}

% Thesis information
\title{Sensitivity of retinal image segmentation on ground truth accuracy}
\titlefin{Silm�npohjakuvien segmentoinnin herkkyys pohjamerkint�jen tarkkuudelle}
\author{Teemu Huovinen}

\Major{Degree Program in Computer Science}
\Majorfin{Tietotekniikan koulutusohjelma}

\Keywords{eye fundus, image segmentation, sensitivity analysis, ground truth}

\Keywordsfin{silm�npohja, segmentointi, herkkyysanalyysi, pohjamerkint�}

% For a single supervisor, \Supervisor{N.N.}
% For multiple supervisors, \Supervisors{N.N.\\ K.K.}, that is,
% use \\ to separate names.
% the same with \Examiner{} or \Examiners{}
\Supervisor{Lasse Lensu D.Sc. (Tech.)}
\Examiners{Lasse Lensu D.Sc. (Tech.)}

% The examiners for the Finnish abstract only.
\Tarkastajat{TkT Lasse Lensu}

% date of topic accepted in the council
\AcceptDate{January 1\textsuperscript{st}, 2014}
% date of signature
\SignDate{July 3\textsuperscript{rd}}
% Year in abstract pages
\Year{2014}

% Thesis statistics: figure, table and appendix counts, for abstracts
\addtostats{, 13 figures, 1 table, and 2 appendices}
\addtostatsfin{, 13 kuvaa, 1 taulukko ja 2 liitett�}

\begin{document}
\selectlanguage{english}

\maketitle
\newpage

% These name-definitions must be after Babel langugage change
% commands, as they redefine these.
\renewcommand\refname{REFERENCES}
\renewcommand\contentsname{CONTENTS}

\pagestyle{masters}
\newpage

\tableofcontents

% space between paragraphs
\setlength{\parskip}{3ex}


\section{INTRODUCTION}

\subsection{Background}

%Diabetes has established itself as a seemingly permanent, worldwide global problem. 
The growing amount of diabetes patients and (arguably) more importantly the estimated amount of undiagnosed patients 
motivate the research for an effective mass screening method for monitoring and early detection of diabetes. The most common
complication of diabetes, diabetic retinopathy, causes abnormalities in the eye, and detecting these abnormalities in the eye fundus
is a promising mass screening method. ~\cite{Kauppi:2010} When developing a method for detecting these abnormalities, handmade annotations of the objects
in the image are used as a ground truth to train classifiers and to evaluate the results. In eye fundus image segmentation research,
ground truths are usually done by medical experts of the field (ophthalmologists) marking these abnormalities, such as exudates.

Optimal ground truth would be a pixel-accurate binary representation of the abnormalities, but as ground truths are
done by a human hand, such accuracy is not possible. Because the marking of an accurate ground truth takes a good
amount of time and patience, it is often necessary to have to settle for rough markings of the present abnormalities. Clusters of
exudates are circled, rather than each small finding specified separately.

\subsection{Objectives and Restrictions}

This thesis will address two main questions in its research:
\vspace{-5mm}
\begin{itemize} \itemsep1pt \parskip0pt
\parsep0pt
  \item How will inaccurate ground truth affect features and segmentation methods?
  \item How will other features than colour perform?
\end{itemize}
\vspace{-5mm}

% The objective of this thesis is to evaluate how big of an impact inaccurate ground truth has on various image features
% and segmentation methods. 

This thesis focuses on exudate detection in terms of segmentation performance, and Bristol database is used as it is 
readily available and has accurate ground truths of exudates. To enable comparison between features, colour, edge and
texture features are used. Blood vessel detection is explored only to create a mask for them. A rough method for optic
disk detection is also implemented as a preprocessing step for masking reasons. Structure of the eye fundus is briefly explored for context.

Both supervised and unsupervised segmentation methods are used. In supervised methods, ground truths are used to label
observations as either exudate or background. In unsupervised methods, ground truth is used to evaluate segmentation
results. Best parameters for each method are chosen based on their performance.

\subsection{Structure of the Thesis}

Section 2 takes a look at the different features
of the eye fundus, and how they are relevant in this thesis.
It also explains the theory behind the applied pre-processing
and segmentation methods.
Section 3 details how sensitivity analysis is 
done in this thesis, and also explains the used evaluation methods.
Section 4 describes the experiments in detail,
and presents the results for each experiment.
Section 5 sums up and interprets the results, and
discusses the impact of this thesis and possible future work this thesis
might invoke.


\section{LITERATURE REVIEW}

During this thesis, the following books will be used:

\vspace{-5mm}
\begin{itemize} \itemsep1pt \parskip0pt
\parsep0pt
  \item Kauppi: Eye Fundus Image Analysis For Automatic Detection of Diabetic Retinopathy (doctoral thesis)
  \item Jain: Fundamentals of Digital Image Processing
\end{itemize}
\vspace{-5mm}

Scientific articles will be sought out from databases dedicated to publishing them, such as:

\vspace{-5mm}
\begin{itemize} \itemsep1pt \parskip0pt
 \parsep0pt
  \item IEEE Xplore
  \item Science Direct
  \item Springer
\end{itemize}
\vspace{-5mm}

Most literature used in this thesis will be scientific articles concerning methods of eye fundus segmentation,
focusing on exudate detection. A medical publication describing the structure of the eye will also be sought out.


\section{METHODS OF STUDY}

Research was done by first evaluating possible segmentation methods, both supervised and unsupervised. From these,
the methods best suited for our purpose were chosen for implementation. Sensitivity analysis of segmentation methods was done by evaluating the 
performance of these methods using ground truths of varying accuracy. This also indirectly sheds some light on sensitivity of specific features.

The tests and computations needed for this thesis were conducted during the summer for the Laboratory of 
Machine Vision and Pattern Recognition. During the summer, an image segmentation pipeline was constructed
for evaluation of segmentation methods using varying image and ground truth data. This thesis will be written 
based on the results and observations gathered from these tests.

\section{SCHEDULE}

During the summer, the segmentation pipeline was constructed and initial results were obtained with the following schedule:

\vspace{-5mm}
\begin{itemize} \itemsep1pt \parskip0pt
 \parsep0pt
  \item In June, literature review and the assessment of possible segmentation methods were conducted.
  \item In July, the construction of the image segmentation pipeline was focused on.
  \item In August, the pipeline was used to conduct sensitivity analysis on segmentation methods and specific features.
\end{itemize}
\vspace{-5mm}

Parts of the thesis were also written during the summer.
In the following semester of September to December, the missing parts of the thesis will be completed.

% \clearpage

% Bibliography
%
%% This must be here, not in preamble, if you want it to work
\addcontentsline{toc}{section}{REFERENCES}
\bibliography{resources/thesis}

\end{document}
